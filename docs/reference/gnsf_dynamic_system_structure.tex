\documentclass{report}

\usepackage{amsmath}
\usepackage{amssymb}
\usepackage{color}

\newcommand {\matr}[2]{\left[\begin{array}{#1}#2\end{array}\right]}

\newcommand{\R}{{\mathbb{R}}}
    
\newcommand{\ind}[1]{{_{\text{#1}}}}
\newcommand{\diag}[1]{\text{diag}\left( {#1} \right)}
\newcommand{\upind}[1]{\ensuremath{{^{[\text{#1}]}}}}
\newcommand{\uind}[1]{^{[#1]}}
\newcommand{\defI}[1]{\textit{#1}}
\newcommand{\nxone}{{n_{x_1}}}
\newcommand{\nxtwo}{{n_{x_2}}}
\newcommand{\nx}{\ensuremath{n_x}}
%	\newcommand{\nxone}{{n_{x^{\text{[1]}}}}}
%	\newcommand{\nxtwo}{{n_{x^{\text{[2]}}}}}
\newcommand{\nz}{{n_{z}}}
\newcommand{\nout}{n_{\text{out}}}
\newcommand{\nin}{n_{\text{in}}}
\newcommand{\ninx}{n_{y}}
\newcommand{\ninu}{n_{\hat{u}}}
%\newcommand{\dtAnf}[1]{\glqq#1\grqq}
\newcommand{\total}[0]{\text{d}}
\newcommand{\DD}[0]{\text{D}}
\newcommand{\dtotal}[2]{\frac{\text{d} {#1}}{\text{d} {#2} }}
\newcommand{\dpartial}[2]{\frac{\partial {#1}}{\partial {#2} }}

\begin{document}

\chapter*{Generalized Nonlinear Static Feedback Structure}
\newcommand{\axdot}{E_{11}}
\newcommand{\ax}{A_{1}}
\newcommand{\au}{B_{1}}
\newcommand{\aphi}{C_{1}}
\newcommand{\az}{E_{12}}
\newcommand{\bx}{A_{2}}
\newcommand{\bu}{B_{2}}
\newcommand{\bg}{C_{2}}
\newcommand{\bz}{E_{22}}
\newcommand{\bxdot}{E_{21}}
\newcommand{\lo}{A_{\text{LO}}}
\newcommand{\lx}{L_x}
\newcommand{\lxdot}{L_{\dot{x}}}
\newcommand{\lu}{L_u}
\newcommand{\lz}{L_z}

The following dynamic system structure is called Generalized Nonlinear Static Feedback structure (GNSF), as it consists of a nonlinear static feedback (NSF) coupled to an index-1 differential algebraic equation (DAE) and a linear output system (LOS). It reads as:

\begin{align*}
E \matr{c}{\dot{x}\uind{1} \\ z} & = A x\uind{1} + B u + C \phi(\lxdot\dot{x}\uind{1}+ \lx x\uind{1} + \lz z + \lu u)  	\\
%	&= \bx x\uind{1} + \bu u + \bg g( x\uind{1}, z, u) 	\label{eq:dyn_nlsf_dae_b} \\
\dot{x}\uind{2} &= \lo x\uind{2} + f(\dot{x}\uind{1}, x\uind{1}, z,u) 
\end{align*}

where we denote the states as $ x = \left[ {x\uind{1}}^\top, {x\uind{2}}^\top\right]^\top \in \R^{\nx} $, $ x\uind{1} \in \R^\nxone, x\uind{2}\in\R^\nxtwo $, the algebraic variables $ z \in \R^\nz $, the controls $ u \in \R^{n_u} $, the (nonlinear) functions $ \phi: \R^{\ninx + \ninu} \rightarrow \R^{\nout}, f: \R^{2\nxone + \nz + n_u} \rightarrow \R^{\nxtwo} $, which will be referred to as \defI{nonlinearity} and \defI{linear output function} respectively. Furthermore we use the model matrices $ \lxdot,\lx \in \R^{\ninx \times \nxone} $, $ \lz \in \R^{\ninx \times \nz} $, $ \lu \in \R^{\ninu \times n_u} $ (\defI{linear input matrices}) and the matrices $ A\in \R^{(\nxone + \nz) \times \nxone}$  $ B \in \R^{(\nxone + \nz) \times n_u},
C \in \R^{(\nxone + \nz) \times \nout},$  $E \in \R^{(\nxone + \nz) \times (\nxone + \nz)} $.

\end{document}
